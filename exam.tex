% for appropriate date format, change babel language in .cls
\documentclass[english, midterm]{iexam} %1: slovene, english; 2: exam, midterm
\begin{document}

% change according to your data
%-----------------------------------------------------------
% name of the subject
\subject{Portali in sistemi znanja}
% academic year
\academicyear{2022-2023}
% study program
\programme{ITK}
% id of exam
\examid{1}
% location of exam
\location{A-306}
% date of exam
\newdate{examdate}{18}{11}{2022}
% acronym of subject
\subacronym{PISZ}
% writing time in minutes
\writingtimeminutes{90}
% instructions
\instructionstext{
    \begin{enumerate}
        \item The exam consists of 10 questions.\\
        \item The use of a calculator is not allowed.\\
        \item Write only what the question asks for.\\
        \item You do not get minus points for wrong answers.\\
    \end{enumerate}
}

%\instructionstext{
    %\begin{enumerate}
        %\item Izpit je sestavljen iz 10 vprašanj.\\
        %\item Uporaba kalkulatorja ni dovoljena. \\
        %\item Zapišite le tisto, kar zahteva vprašanje. \\
        %\item Napačni odgovori ne prinašajo minus točk. \\
    %\end{enumerate}
%}
%-----------------------------------------------------------

% create first page
%-----------------------------------------------------------
\makefirstpage
%-----------------------------------------------------------
\makenextheaders
%-----------------------------------------------------------
% questions of the exam
% sytnax
% \qncounter{No. of question}
% \question{What do you want to ask?}{No. of points}{conjugation}{vspace in cm}
%-----------------------------------------------------------

\qncounter{1}
\question{Naštejte nekaj slabosti standardizacije podatkov.}{3}{s}{0}

\qncounter{2}
\question{Zapišite glavne značilnosti inteligentnih sistemov.}{2}{s}{1}

\qncounter{3}
\question{Naštejte vsaj 5 tehnik priprave podatkov in zapišite njihove značilnosti in za kakšen namen se določena tehnika uporablja?}{6}{s}{0}

\begin{enumerate}[i]
    \alinea{Zakaj je pomembno izvesti pripravo podatkov preden začnemo učiti klasifikator?}{4}{s}{1}
    \alinea{Kaj pomeni termin \textbf{podatkovna množica} in kako je sestavljena?}{1}{}{1}
\end{enumerate}

\qncounter{4}
\question{Obkroži črko pred pravilnim odgovorom}{1}{}{0}
\begin{enumerate}
    \item Testni
    \item Pravilen
    \item Nepravilen
\end{enumerate}

\qncounter{5}
\longquestion{Naštejte nekaj slabosti standardizacije podatkov.Naštejte nekaj slabosti standardizacije podatkov.Naštejte nekaj slabosti standardizacije podatkov.Naštejte nekaj slabosti standardizacije podatkov.Naštejte nekaj slabosti standardizacije podatkov.Naštejte nekaj slabosti standardizacije podatkov.}{2}{s}{1}

\end{document}
